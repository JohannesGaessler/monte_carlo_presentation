\section{Simple Benchmark}
To better illustrate the benefits of GPUs we did a few simple benchmarks:
we timed the MC calculation of $\pi$ on a GPU using the Tensorflow (v2.5.0-2) package and compared it
to an equivalent implementation using the NumPy (v1.20.3-1) package.
All benchmarks were performed on a Linux platform using a Ryzen 3700X CPU and an NVIDIA GTX 1070 GPU.
The CPU and GPU are roughly equal in terms of price and release date.
\subsection{Hit-And-Miss MC}
The calculation of $\pi$ using hit-and-miss MC can be implemented with the following Python code:
\begin{lstlisting}
@tf.function
def mc_tf(sample_size):
    rand_xy = tf.random.uniform(
	(sample_size, 2))

    # Map:
    in_circle = tf.square(rand_xy[:, 0]) \
	+ tf.square(rand_xy[:, 1]) < 1.0

    # Reduce:
    mc_estimate = tf.math.count_nonzero(
	in_circle) / sample_size

    return 4 * mc_estimate
\end{lstlisting}
Notably the entire calculation can be expressed as a combination of map and reduce operations,
implying that it can be efficiently parallelized on GPUs.
\begin{figure*}
	\centering
	\includegraphics[width=\linewidth]{pi_hit_and_miss_benchmark.png}
	\caption{
		Benchmark of hit-and-miss MC.
		Runtime with NumPy is proportional to sample size.
		Runtime with Tensorflow is constant for $N <= 10^7$.
		Runtime with Tensorflow is proportional to sample size for $N >= 10^9$
		(assuming Tensorflow is already initialized).
	}
	\label{fig:pi_hit_and_miss_benchmark}
\end{figure*}
Figure \ref{fig:pi_hit_and_miss_benchmark} shows the result of the benchmark.
The runtime with NumPy (single CPU core) is proportional to sample size, indicating negligible overhead.
The runtime with Tensorflow (GPU) is flat for small sample sizes and proportional to sample size for large sample sizes,
indicating significant overhead.
For small sample sizes NumPy is much faster than Tensorflow, for large sample sizes Tensorflow is much faster than NumPy.

A large part of the GPU overhead is due to the initialization of the Tensorflow library which is several gigabytes in size.
However, some overhead still remains when this is accounted for.
The reason for this overhead is the memory management described in section \ref{sec:programming}.
Compared to computation memory accesses are very slow.
Memory transfers between system memory and GPU memory are even slower.
To execute the above code on a GPU the generated kernel needs to first be transferred to GPU memory.
After executing code the result needs to be transferred back to system memory.
For small sample sizes the runtime is entirely dominated by the memory transfers,
resulting in a runtime that is essentially constant.
For large sample sizes the time needed for memory transfers becomes negligible resulting in a
runtime that is proportional to sample size.
\subsection{Crude MC}
The calculation of $\pi$ can be implemented with the following Python code:
\begin{lstlisting}
@tf.function
def mc_tf(sample_size):
    rand_x = tf.random.uniform(
	(sample_size,))

    # Map:
    function_values = 1.0 - tf.math.sqrt(
	1.0 - tf.math.square(rand_x))

    # Reduce:
    mc_estimate = tf.reduce_mean(
	function_values)

    return 4 * mc_estimate
\end{lstlisting}
Again, the entire calculation can be expressed as a combination of map and reduce operations.
\begin{figure*}
	\includegraphics[width=\linewidth]{pi_crude_benchmark.png}
	\caption{
		Benchmark of crude MC.
		Runtime with NumPy is proportional to sample size.
		Runtime with Tensorflow (GPU) is constant for $N <= 10^7$.
		Runtime with Tensorflow (CPU and GPU) is proportional to sample size for $N >= 10^9$
		(assuming Tensorflow is already initialized).
	}
	\label{fig:pi_crude_benchmark}
\end{figure*}
Figure \ref{fig:pi_crude_benchmark} shows the result of the benchmark.
The results for NumPy and Tensorflow (GPU) are essentially the same as for the hit-and-miss implementation.
Additionally to Tensorflow (GPU) we also benchmarked Tensorflow (CPU) to investigate the runtime
of a multithreaded CPU implementation.
Just like with Tensorflow (GPU) there is significant overhead that becomes negligible for large sample sizes.
The overhead is due to the cost associated with thread creation.
At least in this benchmark Tensorflow (GPU) was always faster than Tensorflow (CPU), particularly for large sample sizes.
